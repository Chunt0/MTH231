\documentclass{article}
\usepackage[margin=.9in]{geometry}
\usepackage[dvipsnames]{xcolor}
\usepackage{amsmath}
\usepackage{amssymb}
\usepackage{amsthm}
\usepackage{mathrsfs}
\usepackage{tikz}
\usepackage{csquotes}
\usepackage{float}
\usepackage[normalem]{ulem}
\newtheorem*{claim}{Claim}
\newtheorem*{poof}{Proof}
\title{HW 6}
\author{Christopher Hunt}
\date{}
\usepackage{graphicx} 
\usepackage{fancyhdr}

\begin{document}
\pagestyle{fancy}
\fancyhf{}
\rfoot{MTH 231}
\lfoot{Christopher Hunt}
\lhead{HW 6}
\rhead{\thepage}
\maketitle

\section*{1.1 - 14. The number 735000 factors as $2^3*3^1*5^4*7^2$. How many divisors does it have?}
From the problem description we are given the prime factors of the number $735000$ which are $2^3*3^1*5^4*7^2$.
$$$$ 
There are four ways $2$ can divide the value:
$$1,2,4,8$$
There are two ways $3$ can divide the value:
$$1,3$$
There are five ways $5$ can divide the value:
$$1,5,25,125,625$$
And there are three ways $7$ can divide the value:
$$1,7,49$$
By multiplying the number of ways each prime factor can divide the value we can deduce the number of possible divisors for the number 735000.
$$4*2*5*3=120$$
There are $120$ divisors. 

\newpage

\section*{2.2 - 13. If you have enough toothpicks, you can make a large triangular grid. Below, are the triangular grids of size 1 and of size 2. The size 1 grid requires 3 toothpicks, the size 2 grid requires 9 toothpicks.}
\begin{figure}[H]
    \includegraphics*[width=1\linewidth]{2_2_13.png}
\end{figure}
\subsection*{a) Let $t_n$ be the number of toothpicks required to make a size $n$ triangular grid. Write out the first 5 terms of the sequence $t_1,t_2,...$.}
$$t_1 = 3 \quad t_2 = 9 \quad t_3 = 18 \quad t_4 = 30 \quad t_5 = 45$$
\subsection*{b) Find the recursive definition for the sequence. Explain why you are correct.}
To find the recursive definition let's see if there is an a pattern between the different values $t_n$
\begin{align*}
    t_2-t_1 &= 6 \\
    t_3-t_2 &= 9 \\
    t_4-t_3 &= 12 \\
    t_5-t_4 &= 15
\end{align*}
$$(6,9,12,15)$$
This is an algebraic sequence. The recursive definition can be defined as follows:
$$t_n = t_{n-1} + 3n$$
\subsection*{c) Is the sequence arithmetic or geometric? If not, is it the sequence of partial sums of an arithmetic or geometric sequence? Explain why your answer is correct.}
This sequence is not arithmetic, since the difference between each term is not constant, and it is not geometric, since the ratio between successive terms is not constant. The sequence defined by the difference between successive terms, however, is an arithmetic sequence. This means $t_n$ is the sequence of partial sums of the sequence (6, 9, 12, 15, ...).
\subsection*{d) Use you results from part (c) to find a closed formula for the sequence. Show your work.}
Since our sequence is the sequence of partial sum of the arithmetic sequence (6, 9, 12, 15, ...), the closed definition will be some multiple of closed definition for the triangle numbers. Upon inspection that multiple value is 3. The closed definition can be written as follows:
$$t_n = 3\Big(\frac{n(n+1)}{2}\Big)$$
\newpage

\section*{SQ-8. Determine the number of five-card poker hands that contain three queens. How many of them contain, in addition to the three queens, another pair of cards?}

To find the number of five-card poker hands that contain three queens we can consider two different choice cases. In the first case, out of the four queens available in the deck 3 are chosen, then from the remaining 48 cards we choose two.
$$\binom{4}{3}*\binom{48}{2}$$
In the second case, we choose the frist three queens, then from the remaining 48 we choose 1 and then we get the fourth queen free.
$$\binom{4}{3}*\binom{48}{1}*\binom{1}{1}$$
To find the total number of combinations, we add these two values together:
$$\binom{4}{3}*\binom{48}{2}+\binom{4}{3}*\binom{48}{1}\binom{1}{1} = 4704$$
There are $4704$ possible poker hands that contain at least three queens.
$$$$
Next, if we want to find the possible hands with three queens and two pairs we must first choose three of the four queens, then draw one of the remaining forty-eight, then from the remaining three suits of that previous card, draw one. This total must then be divided by half since order doesn't matter and we need to remove the double counted pairs.
$$\frac{\binom{4}{3}*\binom{48}{1}*\binom{3}{1}}{2!} = 288$$
There are $288$ possible full house poker hands with three queens!

\newpage
\section*{SQ-9. How many permutations of ${1,2,3,4,5}$ leave exactly 1 element fixed?}
To begin this problem we assume that order matters, let's consider the index of the ordered set corresponds to the value in that position. Now consider the possible ways in which one value is fixed and the others are not. For each scenario there will be one element fixed, there are 5 possible ways of holding one element of a 5 element set in place. The remaining four slots must be filled by values that do not match the index. The total number of possible remaining permutations can be expressed as $4!$. We then want to exclude from that all the permutations where the value matches the index. We start by selecting one element from the set of 4 remaining numbers. Once we have fixed that element we are left with three remaining elements that can be permuted in $3!$ ways. This count includes some elements that are in their original index, to exclude them we need to subtract the cases where two elements are fixed. So we have four choose two choices for determining which two elements to fix, and for each pair, the remaining elements can be permuted in $2!$ ways. This subtraction is excessive, as it also removes the cases where three elements are fixed. So we need to add back in the permutations where three elements are fixed. There are four choose three ways, times $1!$ remaining indices. Finally we need to exclude the case where all four elements are fixed, which corresponds to four choose four, leaving $0!$ permutations left. This then gives us:

$$5*\Big(4! - \Big[\binom{4}{1}3! - \binom{4}{2}2! + \binom{4}{3}1!-\binom{4}{4}0!\Big]\Big) = 45$$
There are 45 total permutations of ${1,2,3,4,5}$ when you leave exactly 1 element fixed.



\end{document}
