\documentclass{article}
\usepackage[margin=.9in]{geometry}
\usepackage[dvipsnames]{xcolor}
\usepackage{amsmath}
\usepackage{amssymb}
\usepackage{amsthm}
\usepackage{tikz}
\usepackage{mathrsfs}
\usepackage{float}
\newtheorem*{claim}{Claim}
\newtheorem*{poof}{Proof}
\title{HW 7}
\author{Christopher Hunt}
\date{}
\usepackage{graphicx} 
\usepackage{fancyhdr}

\begin{document}
\pagestyle{fancy}
\fancyhf{}
\rfoot{MTH 231}
\lfoot{Christopher Hunt}
\lhead{HW 7}
\rhead{\thepage}
\maketitle

\section*{15. Prove that any graph with at least two vertices must have two vertices of the same degree.}


\newpage
\section*{7.The two problems below can be solved using graph coloring. For each problem, represent the situation with a graph, say whether you should be coloring vertices or edges and why, and use the coloring to solve the problem.}
For this problem I will color the edges. Each color of edge represents a day a match is played.

\subsection*{a. Your Quidditch league has 5 teams. You will play a tournament next week in which every team will play every other team once. Each team can play at most one match each day, but there is plenty of time in the day for multiple matches. What is the fewest number of days over which the tournament can take place?}
\subsection*{b.Ten members of Math Club are driving to a math conference in a neighboring state. However, some of these students have dated in the past, and things are still a little awkward. Each student lists which other students they refuse to share a car with; these conflicts are recorded in the table below. What is the fewest number of cars the club needs to make the trip? Do not worry about running out of seats, just avoid the conflicts.}

\begin{center}
\begin{tabular}{|c|c|}
    \hline
    \textbf{Student} & \textbf{Conflicts} \\
    \hline
    A & BEJ \\
    \hline
    B & ADG \\
    \hline
    C & HJ \\
    \hline
    D & BF \\
    \hline
    E & AI \\
    \hline
    F & DJ \\
    \hline
    G & B \\
    \hline
    H & CI \\
    \hline
    I & EHJ \\
    \hline
    J & ACFI \\
    \hline
\end{tabular}
\end{center}


\end{document}