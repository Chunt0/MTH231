\documentclass{article}
\usepackage[margin=.9in]{geometry}
\usepackage{xcolor}
\usepackage{amsmath, amsthm}
\usepackage{amssymb}

\newtheorem*{claim}{Claim}
\newtheorem{lemma}{Lemma}
\newtheorem*{poof}{Proof}

\title{HW 4}
\author{Christopher Hunt}
\date{}

\usepackage{graphicx} 
\usepackage{fancyhdr}

\begin{document}
\pagestyle{fancy}
\fancyhf{}
\rfoot{MTH 231}
\lfoot{Christopher Hunt}
\lhead{HW 4}
\rhead{\thepage}

\maketitle

\section*{17. Give a proof of the statement, “for all $n\in \mathbb{N}$, the number $n^2+n$ is even.”}
Let P(n) be "$n^2 + n$ is even".
\begin{claim}
    $\forall n \in \mathbb{N}$, P($n$)
\end{claim}
\begin{poof}
    Suppose  $n=1$, P($1$) can be written as:
    \begin{align*}
        1^2+1\\
        1+1\\
        2
    \end{align*}
    Since 2 is an even number, by the definition of even numbers, P(1) is true.\\
    
    \noindent Now let us assume, for some natural number $k$, P($k$) is true.
    Now consider P($k+1$):
    \begin{align*}
        (k+1)^2+&k+1\\
        k^2+2k+1+&k+1\\
        (k^2+k)+&2k+2\\
        (k^2+k)+&2(k+1)
    \end{align*}
    Since we know that $k^2+k$ and $2(k+1)$ are both even and the sum of two even numbers is even, P($k+1$) will be even.
    Therefore by the principle of mathematical induction all natural numbers, $n$, the number $n^2+n$ is even.
    \\
    \qed
\end{poof}
\newpage
\section*{22. Suppose that a particular real number 
 has the property that $x+\frac{1}{x}$ is an integer. Prove that $x^n+\frac{1}{x^n}$ is an integer for all natural numbers.}

\begin{claim}
     $x^n+\frac{1}{x^n}$ is an integer for all natural numbers.
     $$\forall n \in \mathbb{N}, x^n +\frac{1}{x^n}\text{ \emph{is an integer}}$$
\end{claim}
\begin{poof}
     Let's begin by considering the base case where n = 1.
    \begin{align*}
         x^1+&\frac{1}{x^1}\\         x+&\frac{1}{x}
    \end{align*}
    Since $x+\frac{1}{x}$ is assumed to be an integer we can state that the claim where n=1, is true.
    \\
    
    \noindent Now let us assume for all integers from 0 to k the claim holds true. Since we know that integers have closure under multiplication if we multiplied the integer $x^k+\frac{1}{x^k}$ by $x+\frac{1}{x}$ the product will be an integer, consider the expansion of this product equals some integer j.
    \begin{align*}
        (x^k+\frac{1}{x^k})(x+\frac{1}{x})=&\,j\\
        x^kx+\frac{x^k}{x}+\frac{x}{x^k}+\frac{1}{x^kx}=&\,j\\
        x^{k+1}+\frac{1}{x^{k+1}}+x^{k-1}+\frac{1}{x^{k-1}}=&\,j\\
    \end{align*}
    Since we are assuming that the claim holds for all values up to k, it will then hold true for $k-1$. We can replace $x^{k-1}+\frac{1}{x^{k-1}}$ for some integer $m$
    \begin{align*}
        x^{k+1}+\frac{1}{x^{k+1}}+x^{k-1}+\frac{1}{x^{k-1}}=&\,j\\
        x^{k+1}+\frac{1}{x^{k+1}}+m=&\,j
    \end{align*}
    Since we know that both j and m are integers and integers are closed under addition the sum of $x^{k+1}+\frac{1}{x^{k+1}}$ must also be an integer. Therefore by strong induction, the claim is true.
    \\
    \qed
\end{poof}

\newpage
\section*{SQ-3. Claim: $\forall n \in \mathbb{N} \text{ such that }F_n < 2^n$. Prove this using induction.}
To prove this first we must prove this lemma:
\begin{lemma}
    \begin{claim}
        All values, are natural numbers. If some $x$ less than some $n$ and some $y$ is less than some $m$, the sum of $x$ and $y$ will be less than the sum of $n$ and $m$.
    \end{claim}
    \begin{poof}
        Let us assume $x$ is less than $n$ and $y$ is less than $m$. This means that there are some positive integer $j$ and $k$ such that $x+j=n$ and $y+k=m$. We can write this equality:
        \begin{align*}
            x+y =& n+m-(j+k)
        \end{align*}
        Since both $j$ and $k$ are positive integers, the sum of $n$ and $m$ will have the value $j+k$ subtracted from it meaning it will be smaller. Therefore, the claim is true.\\
        \qed
    \end{poof}
\end{lemma}
Let's begin our proof by defining the Fibonacci Sequence as follows:
$$\text{Fibonacci Sequence: }F_n+F_{n+1}=F_{n+2}\qquad F_0 = 0 \text{ ad }F_1=1 \qquad 0,1,1,2,3,5...$$
\begin{claim}
    For all natural numbers n, $F_n < 2^n$
\end{claim}
\begin{poof}
    Consider the two base cases where n = 0 and n = 1:
    \begin{align*}
        F_0<\,2^0 \text{ and }& F_1<\,2^1\\
        0<\,1\text{ and }&1<\,2
    \end{align*}
    Zero is less then one and one is less than two the claim holds true for n = 0 and n = 1.
    \\
    \noindent Now assume that the claim holds true for all values from 0 to k. I want to show that the claim will hold true for n=k+1
    \begin{align*}
        F_{k+1}<&\,2^{k+1}\\
        F_{k+1}<&\,2^{k}2^1\\
        F_{k+1}<&\,2^{k+1}2\\
        F_{k+1}<&\,2^{k}+2^{k}\\
    \end{align*}
    By the definition of the Fibonacci Sequence the $F_{k+1}$ value in the sequence will be equal to the sum of the previous two values, $F_{k+1}=F_k+F_{k-1}$. Since we are assuming that $F_k$ and $F_{k-1}$ are both smaller than $2^k$, then their sum will be smaller than the sum of $2_k+2_k$ by lemma 1. This means that $F_{k+1}$ is smaller than $2^k+2^k$. Therefore, by strong induction the claim is true.\\
    \qed
\end{poof}

\end{document}
